\label{key}
\documentclass[12pt, letter paper]{article}
\usepackage[margin=1in]{geometry}
\usepackage{amsmath}
\usepackage{amssymb}
\usepackage{fancyhdr}
\usepackage{amsthm}
\usepackage{enumitem}
\usepackage{mathtools}
\usepackage{tikz}
\usetikzlibrary{automata,positioning}
\author{To, Jean-Francois}

\pagestyle{fancy}
\renewcommand{\headrulewidth}{0pt}
\renewcommand{\footrulewidth}{0pt}

\setlength{\parindent}{0pt}
\fancyhf{}
\lhead{COMP 330 Assignment 6\\ Theory of Computation \\ }
\rhead{To, Jean-Francois \\ McGill ID: 260927482 \\ }
\rfoot{\thepage}

\begin{document}
\begin{enumerate}[label=\textbf{Question \arabic*},align=left]
	\item ~\\
		\begin{enumerate}
			\item 
				This is false. If $T_i$ is a TM that decides  $C_i$ and given a word
				$w$, we would have to input  $w$ to every single $T_i$ to know if $w$ is
				in the set or not. But this is not possible since it would never end.
			\item 
				This is true. Let $T_i$ be the generator of  $C_i$. We generate the
				union as follows: Run $ T_1$ for one step, run $ T_1,T_2$ for two steps,
				and so on. This generates the entire union and hence it is computably
				enumurable.
		\end{enumerate}
\end{enumerate}
\end{document}
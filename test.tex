\label{key}
\documentclass[12pt, letter paper]{article}
\usepackage[margin=1in]{geometry}
\usepackage{amsmath}
\usepackage{amssymb}
\usepackage{fancyhdr}
\usepackage{amsthm}
\usepackage{enumitem}
\usepackage{mathtools}
\usepackage{tikz}
\usetikzlibrary{automata,positioning}
\author{To, Jean-Francois}

\pagestyle{fancy}
\renewcommand{\headrulewidth}{0pt}
\renewcommand{\footrulewidth}{0pt}

\setlength{\parindent}{0pt}
\fancyhf{}
\lhead{COMP 330 Assignment 6\\ Theory of Computation \\ }
\rhead{To, Jean-Francois \\ McGill ID: 260927482 \\ }
\rfoot{\thepage}

\begin{document}
\begin{enumerate}[label=\textbf{Question \arabic*},align=left]
	\item ~\\
		\begin{enumerate}
			\item 
				This is false. If $T_i$ is a TM that decides  $C_i$ and given a word
				$w$, we would have to input  $w$ to every single $T_i$ to know if $w$ is
				in the set or not. But this is not possible since it would never end.
			\item 
				This is true. Let $T_i$ be the generator of  $C_i$. We generate the
				union as follows: Run $ T_1$ for one step, run $ T_1,T_2$ for two steps,
				and so on. This generates the entire union and hence it is computably
				enumurable.
		\end{enumerate}
	\item ~\\
		\begin{enumerate}
			\item By b), It is not CE hence it can't be computable
			\item Suppose that $K$ is CE, then there exists a TM $A$ that accepts
				$K$.We build a TM $B$ for $\overline{L}$ as follows. Given a word $w$,
				add $b$ to it so it becomes $bw$. Then give it as input to $K$. This
				machine $B$ accepts all words in  $\overline{L}$ which contradict our
				assumption. Hence  $K$ is not  CE.
			\item  Suppose that $K$ is co-CE, then there exists a TM $A$ that rejects
				$K$. We build a TM $B$ for  $L$. Given  a word $w$, add $a$ so it
				becomes  $aw$ then input it to  $A$. This machine rejects all words that
				are not in $L$. Hence  $L$ is coCE which is a contradiciton.
		\end{enumerate}
	\item It is decidable. Let $\Gamma$ be the alphabet and  $Q$ be the number of
		states. We count the possible number of configurations that use  $330$
		cells. The head can be at  $331$ different positions  each with $Q$ possible
		states. The number of ways of arranging the letters on  $330$ cells is
		$\Gamma^{330}$. Hence the number of configurations is finite ($\Gamma^{330}
		\cdot 331 \cdot Q$). Knowing this, we
		run the TM longer than the number of configurations. If the TM is still
		using only  $330$ cells, then we know that it is in a loop since a
		configuration was repeated. Reject if that is the case. Otherwise accept.
	\item ~\\
		\begin{enumerate}
			\item We build a new NFA $A$ as follows. Everything except the final
				states of $A$ are the same as the machine that recognizes  $L$. The
				final states of  $A$ are simply the states from which when we input $w$,
				we reach a state that is in  $F$ where  $F$ are the final states in the
				machine that recognizes $L$.
			\item 
				Let $N= \left\{ a^nb^n | n \geq 0 \right\} $. Then
				\[
					L = (N \# \Sigma^*) \cup ( \Sigma^* \# L(G) )
				\] 
				is context free. We sketch the grammar:
				\begin{align*}
					S \to X\#A | A\#Y
				\end{align*}
				where $X$ generates $N$ since $N$ is context free.  $A$ simply
				generates all the words of the alphabet. $Y$ generates  $L(G)$ since
				$L(G)$ is context free. Now we show this language is regular if and only
				if $L(G) = \Sigma^*$. Suppose that $L(G) \neq  \Sigma^*$. Take $w$ such
				that  $w \not \in L(G)$. Consider $L / \#w= N$. If $L$ is regular, then
				so is $L / \#=N$ by $1.$ but $N$ is not regular. Hence  $L$ is not
				regular. Now suppose that $L(G) = \Sigma^*$. Then $L = \Sigma^* \#
				\Sigma^*$. It is easy to build a DFA that accepts this language by
				accepting the word if it contains $\#$. Hence it is regular.
				This proves the claim. Furthermore, we proved in class that determinining if $L(G)=\Sigma^*$ is
				undecidable. Hence determining if $L(G)$ is regular is undecidable as well.

		\end{enumerate}
	\item We first keep track of the time $t$. Next we choose $(x_0,y_0)$ 
		where this corresponds to the starting point of the ship. We also have to choose
		the direction. With all of this, we can compute where the ship would be at
		this time $t$. Since $\mathbb{N}
		\times \mathbb{N} $ is countable (and thus $\mathbb{N}  \times \mathbb{N}
		\times 4$ is countable as well), we can enumerate all such tuples. Hence, we can
		generate every possible combinations and as such, we will eventually hit the
		ship.
\end{enumerate}
\end{document}
